\documentclass[%
	%draft,
	%submission,
	%compressed,
	final, %technote,
	%internal,
	%submitted,
	%inpress,
	%reprint,
	% %titlepage,
	notitlepage,
	%anonymous,
	narroweqnarray,
	inline,
	twoside,
        %invited,
	]{ieee}


\usepackage[latin1]{inputenc}			% idioma
\usepackage[spanish]{babel}
%\usepackage{color}
%\usepackage{colortbl}
%\usepackage{amsmath}
%\usepackage{amsfonts}
%\usepackage{verbatim}

\usepackage{hyperref}				% urls

% -------- PARA FIGURAS
\usepackage{graphicx}
%\usepackage{capt-of}
% -------- FIN FIGURAS

% -------- PARA MONXTRAR EL CODIGO DE MANERA AMENA
\usepackage{courier}
\usepackage{listings}
\lstset{ %
	language=C++,                % choose the language of the code
	basicstyle=\footnotesize\ttfamily,       % the size of the fonts that are used for the code
	numbersep=5pt,                  % how far the line-numbers are from the code
%	backgroundcolor=\color{white},  % choose the background color. You must add \usepackage{color}
	showspaces=false,               % show spaces adding particular underscores
	showstringspaces=false,         % underline spaces within strings
	showtabs=false,                 % show tabs within strings adding particular underscores
	frame=single,           % adds a frame around the code
	tabsize=6,          % sets default tabsize to 2 spaces
	captionpos=b,           % sets the caption-position to bottom
	breaklines=true,        % sets automatic line breaking
	breakatwhitespace=false,    % sets if automatic breaks should only happen at whitespace
}
% -------- FIN CODIGO

\hypersetup{			%sacar los colores horrendos de las ref
	colorlinks=false,
	pdfborder={0 0 0},
}


\begin{document}


\title[Wiretapping de paquetes ARP]{%
	Wiretapping de paquetes ARP: analizando redes locales usando Teor\'ia de la Informaci\'on
}

\author[CIRUELOS, MADDONI, PATAN\'E]{
Gonzalo Ciruelos \authorinfo{G. Ciruelos e-mail: gonzalo.ciruelos@gmail.com},
\and{}Axel Maddonni\authorinfo{A. Maddonni, e-mail: axel.maddonni@gmail.com}%
\and{}y Federico Patan\'e\authorinfo{F. Patan\'e, e-mail: fedepatane20@gmail.com}
}

\journal{Teor\'ia de las Comunicaciones, Departamento de Computaci\'on, Universidad de Buenos Aires}

\firstpage{1}

\maketitle               


\begin{abstract} 
Con el objetivo de comprender diversos aspectos de una red local, nos planeteamos analizar sus paquetes ARP usando herramientas de la teor\'ia de la informaci\'on.
Para lograr eso, modelamos alg\'un aspecto de nuestra red como una fuente de inforamci\'on, de manera de poder analizarla con las herramientas previamente dichas.
Adem\'as, comparamos distintos tipos de redes para poder confirmar o refutar nuestras hip\'otesis generales.
\end{abstract}

\begin{keywords}
ARP, LAN, broadcast, unicast, entrop\'ia, informaci\'on, Ethernet, Wi-Fi
\end{keywords}


% \section{Introducci\'on te\'orica}

\PARstart En este trabajo nos proponemos utilizar herramientas de la Teor\'ia de la Informaci\'on y los paquetes ARP para intentar comprender diferentes redes locales. La pregunta que nos intentaremos responder en este trabajo es ?`Podemos conocer la topolog\'ia de la red y/o sus nodos m\'as importantes?

La explicaci\'on de las fuentes utilizadas vendr\'a m\'as adelante, pero es pertinente presentar el formato de los paquetes ARP. El protocolo de resoluci\'on de direcciones (Address Resolution Protocol) \cite{arp} es un protocolo usado para la resoluci\'on de direcciones de la capa de red a direcciones de la capa de enlace (usualmente MAC), lo que es una funci\'on cr\'itica en redes de m\'ultiple acceso.

Un paquete ARP tiene muchas partes, pero la que m\'as nos interesa es el campo OPER, o sea el campo de Operaci\'on, que nos indica qu\'e funci\'on cumple ese paquete. 
Los distintos valores de OPER en paquete ARP son is-at y who-has. 
Un paquete who-has es un paquete broadcast en el que una computadora pregunta al resto de la red qui\'en es la que tiene una direcci\'on de capa de red (generalmente IP) dada.
La respuesta a ese paquete es un paquete is-at, que es un paquete unicast, enviado desde la computadora con la direcci\'on de capa de red requerida hasta la computadora que emiti\'o el who-has.

Por otro lado, con respecto a la teor\'ia de la informaci\'on, utilizaremos los conceptos de informaci\'on y entrop\'ia cl\'asicos.


% \section{Desarrollo}

\PARstart Como explicamos anteriormente, el trabajo se basará en analizar redes locales capturando paquetes ARP de redes locales y analizandolos utilizando herramientas de la teoría de la información.

Para capturar los paquetes de la red analizada, utilizamos el programa \textsc{Wireshark}. Luego, para postprocesar los datos y computar la inforamación pedida utilizamos la librería de Python \textsc{Scapy}. \textsc{Scapy} es una poderosa herramienta que permite capturar, decodificar, crear y envíar paquetes de manera muy sencilla.

\subsection{Ejercicio 1}

El ejercicio 1 proponía analizar los resultados modelando a los paquetes como una fuente de información muy simple: simplemente distinguir entre paquetes broadcast y paquetes unicast. La fuente consiste en dos símbolos, uno que representa a los paquetes unicast, y otro que representa a los paquetes broadcast.

Esta fuente no nos permitirá conocer muy bien quién es quién en la red, pero nos dará quizás un poco de información sobre lo que está pasando en la red local.

Que la cantidad de mensajes de broadcast sea muy grande, o sea, la información del símbolo que representa a los mensajes broadcast sea muy chica, será lo esperado. Esto se debe a que en una red normal, en la cual podemos escuchar todo lo que está pasando, lo esperable es que los dispositivos se comuniquen unos con otros en vez de estar envíando broadcasts todo el tiempo.

De alguna manera, si vemos pocos paquetes unicast, o sea que la información de este símbolo es muy alta, puede significar dos cosas:

\begin{enumerate}
  \item No tenemos visibilidad total de la red, por ejemplo porque está switcheada, y entonces vemos solo los paquetes unicast dirigidos a nuestro host.
  \item No hay comunicación efectiva entre los hosts de la red, porque los paquetes unicast de alguna manera miden cuanta comunicación de un host a otro está sucediendo.
\end{enumerate}

Nuestra implementación de este ejercicio puede verse en el archivo \texttt{ejercicio1.py}.


\subsection{Ejercicio 2}

El ejercicio 2 proponía que diseñemos una fuente de memoria nula en base a los paquetes ARP, que nos permitiera distinguir a los nodos distinguidos de la red, para alguna definición que demos de nodo distinguido.

Experimentamos con varias fuentes, y terminamos seleccionando que la fuente sea el destino de los paquetes who-has, por las siguientes razones:

\begin{enumerate}
  \item Si estamos analizando una red switcheada o muy subdividida en subredes virtuales (VLANs), entonces lo más probable es que solo recibamos paquetes who-has, dado que estos paquetes se envían en modo broadcast.
    Por esta razón los switches de la red local no los filtran y le llegan a todos los hosts.  Quizás esta diferencia no sea notable en redes inalámbricas, pero en redes cableadas complejas, que generalmente tienen switches, puede hacer una gran diferencia.
  \item Como justificamos anteriormente, vamos a usar paquetes who-has, ahora bien, la pregunta es porqu\'e el destino de esos paquetes. En este caso la respuesta es más obvia: si un host es el destinatario de más paquetes ARP, entonces es más requerido por el resto de los hosts, entonces es más probable que sea un nodo distinguido, como por ejemplo un router.
\end{enumerate}

Siguiendo estos simples preceptos, diseñamos nuestra fuente de información S1. Nuestra implementación de este ejercicio puede verse en el archivo \texttt{ejercicio2.py}.

\subsection{Grafo de la red}

Para los tres experimentos que realizamos, hicimos el grafo de la red que se desprende de los paquetes ARP enviados a lo largo de la captura.

El grafo fue realizado de la manera más natural. Por cada mensaje who-has capturado, el grafo tendrá una arista. Además, esa arista irá del nodo con IP igual a la IP fuente del who-has al nodo con IP destino del who-has.

Por razones de comodidad, solo mostramos aquellos nodos relevantes, es decir, los que tienen información baja con respecto a la fuente de información que S1.

Además, juntamos en uno a todos los hosts que tienen exactamente el mismo conjunto de aristas adyacentes. Esto se indicara con un [X]: si un nodo tiene [X] significa que ahí condensamos X cantidad de hosts, que tienen todos exactamente la misma conectividad que el nodo representado. 

Además, marcamos con un cuadrado aquellos nodos distinguidos según la fuente S1, es decir, aquellos que tienen menos información que la entropía.


\subsection{Conceptos generales}

\subsubsection{Gratuitous ARP}

En todos los experimentos apareció un tipo de paquete ARP llamado Gratuitous ARP, con lo cual nos parece mejor introducirlo al principio para dejar en claro qu\'e es y por qu\'e aparece en todos los experimentos.

Gratuitous ARP puede significar tanto un reply (is-at) como un request (who-has). Gratuito en este caso quiere decir que un request o un reply no es normalmente requerido de acuerdo con la especificacion de ARP (RFC 826) \cite{arp}, pero puede ser usado en algunos casos.
Un request ARP gratuito es un paquete donde la IP source y destination están ambas seteadas a la IP del host que envía el paquete. Además, la MAC destino es la dirección de broadcast \texttt{ff:ff:ff:ff:ff:ff}. Ordinariamente, no habrá respuesta para tal request.

Los ARP gratuitos tienen varias utilidades:

\begin{enumerate}
  \item Pueden ayudar a detectar conflictos de IP. Si un host recibe un paquete ARP que contiene una IP source que coincide con la suya, sabe que hay un conflicto.
  \item Ayuda a actualizar las tablas ARP de los hosts de la red.
  \item Cada vez que una interfaz IP se prende, el driver de la interfaz típicamente envía paquetes ARP gratuitos para precargar las tablas ARP de todos los hosts. Por eso, si un host envía muchos paquetes ARP gratuitos, podemos inferir que algo malo está sucediendo con \'el, por ejemplo que se está reiniciando o que su interfaz IP se reinicia continuamente porque no puede iniciarse correctamente.
\end{enumerate}


\subsubsection{Dirección 169.254.255.255}
[eg an address you end up picking yourself because DHCP didn't work usually picked using some arp to figure out if it belongs to someone else]


% \input{resultados.tex}

% \input{discusion.tex}

% \section{Conclusión}

Para concluir, podemos decir que nuestros experimentos fueron exitosos, dado que nos permitieron detallar y especificar técnicas para endender redes locales a partir de la escucha de paquetes ARP, lo que era el objetivo del trabajo.

Para resumir algunos puntos que fuimos dando a lo largo del trabajo para tener en cuenta al analizar redes usando esta técnica podemos decir:

\begin{enumerate}
  \item Si la fuente S tiene una entropía muy alta, quiere decir que la información del símbolo de los paquetes broadcast es muy similar a la información de los paquetes unicast. Esto significará en general que no tenemos visibilidad total de la red, o que el overhead de los paquetes de control es mucho, y no hay comunicación efectiva entre los hosts.
  \item En general los nodos con información menor con respecto a la fuente S1 son los routers o Default Gateways de la red.
  \item Las redes Wi-Fi son bastante simples y se asemejan a un grafo estrella, si no se tiene en cuenta la IP excepcional 169.254.255.255 que fue explicada en detalle anteriormente.
\end{enumerate}

Si queremos comparar las redes Wi-Fi y la redes ethernet, podemos decir que las primeras (en general) son más fáciles de analizar dado que la visibilidad que tenemos de los paquetes de la red es total, y por ejemplo la fuente S se va a comportar de la forma que esperamos.

Por otro lado, las redes Ethernet, si estan fuertemente switcheadas o subneteadas nos van a dar más sesgadas para el lado de los paquetes broadcast, dato que también podemos obtener de la fuente S1.

Fuera de eso, la diferencia no es muy grande. Es importante notar que fue vital nuestra eleccion de fuente S1 en este caso, dado que si hubiéramos elegido como fuente los paqutes is-at (unicast), la fuente hubiera dado muy mal para los casos Ethernet.


En cuanto al tamaño de las redes, la diferencia en la fuente S es casi nula. Sin embargo, en la fuente S1 se observa que la diferencia de información entre los nodos distinguidos y los nodos no distinguidos tiende a ser mayor en las redes más grandes.

La explicación de este fenómeno es realmente simple. A más nodos tiene una red, más nodos van a estar accediendo al router (o al nodo distinguido, sea lo que sea), con lo que va a haber muchos request ARP (who-has) a ese nodo, con lo que la información va a ser muy pequeña. Por otro lado, en redes más pequeñas, la información de los nodos distinguidos también va a ser pequeña, pero menos, dado que la cantidad de requests ARP que recibe va a ser menor en proporción.


Comparar la información de los símbolos de la fuente S1 con la entropía de la fuente S1 fue vital para encontrar los nodos distinguidos. Esto, como explicamos en la introducción, se justifica teóricamente dado que los nodos distinguidos van a ser más comunes como símbolos de la fuente S1 que el resto, con lo cual su información va a ser menor. En particular, un muy buen umbral que se puede elegir para decidir si un símbolo tiene menor información de la esperada es la entropía, dado
que la entropía es literalmente la esperanza de la información.

Una cosa importante para resaltar es que esperaremos que a más nodos haya en nuestra red, más grande sea la entropía. Esto se debe a que si hay muchos hosts en la red, esperamos que sean hosts comunes tipo PC, que en general reciben pocas requests ARP, con lo cual la información de su símbolo asociado va a ser muy alta, provocando que la esperanza de la información (es decir, la entropía), aumente. Por lo tanto, yendo hacia el lado contrario del argumento de recién, esperamos que
si la entropía es alta, entonces la cantidad de nodos no distinguidos sea mayor.
Esto es importante tenerlo en cuenta porque es un dato muy importante que podemos obtener de la red con solo saber la entropía de la fuente S1.

Por último, en general observamos que cuanto más alta es la entropía más nodos distinguidos hay. Esto tiene dos explicaciones posibles, que se complementan una con la otra.
Primero, la explicación numérica: como para nosotros un nodo es distinguido si la información de su símbolo asociado es menor que la entropía, si la entropía es más grande entonces es más probable (por probabilidad) que un símbolo cualquiera tenga menor información que el valor de la entropía.
Segundo, por un tema funcional, si en una red hay muchos nodos, es probable que esa red tenga algún tipo de subestructura y no tenga un único router adentro, si no que tenga varios. Por esa razón, si la entropía es más grande, esperamos que haya más nodos en general, por lo tanto más routers, luego más nodos distinguidos.




%----------------------------------------------------------------------

\begin{thebibliography}{1}

%\bibitem{enunciado}
%C\'atedra de Teor\'ia de las Comunicaciones\\
%\newblock {\em Primer trabajo pr\'actico}\\
%\newblock Primer cuatrimestre $2013$

%\bibitem{arp}
%RFC 826 - Ethernet Address Resolution Protocol\\
%\url{http://tools.ietf.org/html/rfc826}\\
%\newblock C. Plummer $1982$

%\bibitem{conflicto}
%RFC 5227 - IPv4 Address Conflict Detection\\
%\url{http://tools.ietf.org/html/rfc3927}\\
%\newblock S. Cheshire $2008$

%\bibitem{link}
%RFC 3927 - Configuration of IPv4 Link-Local Addresses\\
%\url{http://tools.ietf.org/html/rfc3927}\\
%\newblock Cheshire, et al. $2005$

%\bibitem{patente}
%Method and apparatus for detecting a router that improperly responds to ARP requests\\
%\newblock US 7729292 B2\\
%\url{http://www.google.com/patents/US7729292}\\
%\newblock Stuart D. Cheshire y Joshua V. Graessley

\bibitem{scapy}
\url{http://www.secdev.org/projects/scapy}

%\bibitem{tcp}
%\url{http://www.tcpdump.org}

\end{thebibliography}


\end{document}

