\section{Experimento 1}

El primer experimento que presentamos en el trabajo fue realizado en una cafetería (Starbucks), por la tarde, mientras había aproximadamente unas 5 personas a la vista usando dispositivos móviles. La captura de paquetes duró casi una hora.

\subsection{Resultados}

Primero, la fuente S. Nuestra hipótesis sobre esta fuente, basándonos en lo que dijimos anteriormente en el trabajo, es que el símbolo de las


\begin{figure}[H]
  \centering
  \includegraphics[width=8.5cm]{exp_starbucks/grafico1.pdf}
  \caption{\normalfont }
\end{figure}

\begin{figure}[H]
  \centering
  \includegraphics[width=8.5cm]{exp_starbucks/grafico2.pdf}
  \caption{  \normalfont Grafo de conectividad de la red, inferido de los paquetes who-has. Para ver con mayor detalle, se puede hacer zoom-in en el pdf. }
\end{figure}

\begin{figure}[H]
  \centering
  \includegraphics[width=8.5cm]{exp_starbucks/grafico3.pdf}
  \caption{ \normalfont Información de los símbolos de la fuente S1: solamente los nodos con menor información son representados.}
\end{figure}

\subsection{Discusión}

